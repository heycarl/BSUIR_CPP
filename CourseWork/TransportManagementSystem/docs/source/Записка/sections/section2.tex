\section{Обзор литературы}
Для дальнейшего понимания работы информационной системы необходимо пояснить некоторые термины, которые будут использоваться.
\begin{definition}
    Информационная система (ИС) — система, предназначенная для хранения, поиска и обработки информации, и соответствующие организационные ресурсы (человеческие, технические, финансовые и т. д.), которые обеспечивают и распространяют информацию.
\end{definition}
\begin{definition}
    Общественный транспорт — разновидность пассажирского транспорта как отрасли, предоставляющей услуги по перевозке людей по маршрутам, которые перевозчик заранее устанавливает, доводя до общего сведения способ доставки (транспортное средство), размер и форму оплаты, гарантируя регулярность (повторяемость движения по завершении производственного цикла перевозки), а также неизменяемость маршрута по требованию пассажиров. 
\end{definition}
\begin{definition}
    Маршрут — путь следования объекта, учитывающий направление движения относительно географических ориентиров или координат, с указанием начальной, конечной и промежуточных точек, в случае их наличия. 
\end{definition}
\begin{definition}
    Билет —  документ, удостоверяющий наличие некоего права у какого-либо определённого лица или у предъявителя билета. Действие билета может распространяться на конкретное время или не иметь сроков. 
\end{definition}
\begin{definition}
    Учётная запись — хранимая в компьютерной системе совокупность данных о пользователе, необходимая для его опознавания (аутентификации) и предоставления доступа к его личным данным и настройкам.
\end{definition}
\begin{definition}
    Аутентификация — процедура проверки подлинности пользователя путём сравнения введённого им пароля (для указанного логина) с паролем, сохранённым в базе данных пользовательских логинов.
\end{definition}