\csection{Заключение}

	В данном разделе подведены итоги разработки информационной системы городского автотранспорта.
В ходе разработки и тестирования системы было проверено, что в ней выполняются все ее необходимые функции:
\begin{enumerate}
    \item Создавать учетную запись администратора
    \item Создавать учетную запись пассажира
    \item Создавать нового водителя
    \item Просмотр имеющихся водителей
    \item Редактирование водителя
    \item Создание нового транспортного средства (\textit{ТС})
    \item Просмотр имеющихся транспортных средств
    \item Создание остановочного пункта
    \item Просмотр имеющихся остановочных пунктов
    \item Редактирование остановочного пункта
    \item Создание нового маршута
    \item Добавление остановочного пункта в маршрут
    \item Просмотр статистики маршута
    \item Просмотр информации о маршруте
    \item Покупка проездных билетов
    \item Просмотр количества оставшихся поездок
    \item Входить в транспортное средство
    \item Просмотр времени прибытия ТС на остановочном пункте
\end{enumerate}

Для реализации системы были использованы библиотеки по работе с сериализацией (\textit{Boost}) и стандартные алгоритмы языка С++.
В ходе выполнения курсовой работы были получены навыки работы с языком C++, проработке архитектуры приложения, а так же закреплены навыки работы с объектно-ориентированным программированием. В проекте активно используются основные концепции ООП.
В дальнейшем данную информационную систему можно будет усовершенствовать и добавить больший функционал, к примеру шифрование пароля, создание API-endpoint'ов для сторонних интерграций и создания \textit{GUI}.
Работа над курсовым проектом помогла развить навык конструирования объектно-ориенти-
рованных программ, навык работы со сторонними библиотеками, настройки сборки проектов на языке C++ и работу над архитектурой. 