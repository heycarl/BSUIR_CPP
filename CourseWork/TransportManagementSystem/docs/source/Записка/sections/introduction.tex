\csection{Введение}

С++ — компилируемый строго типизированный язык программирования общего назначения.
Наибольшее внимание уделено поддержке\\объектно-ориентированного программирования. Почему объектно-ориенти-рованный подход к программированию стал приоритетным при разработке большинства программных проектов? ООП предлагает новый мощный способ решения проблемы сложности программ. Вместо чтобы рассматривать программу как набор последовательно выполняемых инструкций, в ООП программа представляется в виде совокупности объектов, обладающих сходными свойствами и набором действий, которые можно с ними производить. 

Синтаксис C++ унаследован от языка C\cite{straus}. Одним из принципов разработки было сохранение совместимости с C. Тем не менее, C++ не является в строгом смысле надмножеством C. Множество программ, которые могут одинаково успешно транслироваться как компиляторами C, так и компиляторами C++, довольно велико, но не включает все возможные программы на C. Нововведениями C++ в сравнении с C являются:
\begin{itemize}
    \item поддержка объектно-ориентированного программирования через \\классы. C++ предоставляет все четыре возможности ООП — абстракцию, инкапсуляцию, наследование (в том числе и множественное) \\и полиморфизм\cite{kova}
    \item поддержка обобщённого программирования через шаблоны
    \item стандартная библиотека C++ состоит из стандартной библиотеки C и библиотеки шаблонов (Standard Template Library, STL), которая предоставляет обширный набор обобщенных контейнеров и алгоритмов
    \item дополнительные типы данных
    \item обработка исключений
    \item виртуальные функции
    \item перегрузка (overloading) операторов
    \item перегрузка имён функций
    \item ссылки и операторы управления свободно распределяемой памятью
\end{itemize}
Данная программа предназначена для информатизации в сфере пассажирских перевозок. При бурном развитии общественного транспорта и глобальной инфморматизаци мира подобный тип программ становится все более актуален.