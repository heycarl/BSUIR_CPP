\section{Структура входных и выходных данных}
В качестве входных файлов в программе используются файлы сохраненых списков объектов. Такими объектами, подлежащими сохранению являются:
\begin{itemize}
    \item Проездные билеты
    \item Транспортные средства
    \item Остановочные пункты
    \item Водители
    \item Маршуты
    \item Учетные записи пользователей
\end{itemize}

\subsubsection{Файл с проездными билетами}
Внутри себя хранит следующую информацию:
\begin{itemize}
    \item Уникальный идентификатор (\textit{UID}) 
    \item Количество поездок
\end{itemize}

\subsubsection{Файл с транспортными средствами}
Внутри себя хранит следующую информацию:
\begin{itemize}
    \item UID
    \item Регистрационный знак
    \item Тип ТС
    \item Вместительность
\end{itemize}

\subsubsection{Файл с остановочными пунктами}
Внутри себя хранит следующую информацию:
\begin{itemize}
    \item UID
    \item Название
    \item Координаты
\end{itemize}

\subsubsection{Файл с водителями}
Внутри себя хранит следующую информацию:
\begin{itemize}
    \item UID
    \item ФИО
    \item Дату рождения
    \item Тип водительской лицензии
    \item Дату истечения лицензии
\end{itemize}

\subsubsection{Файл с маршутами}
Внутри себя хранит следующую информацию:
\begin{itemize}
    \item UID
    \item Список остановочных пунктов
    \item Точку отправления и прибытия
    \item UID водителя, выполняющего рейс
    \item UID ТС, выполняющего рейс
    \item 
\end{itemize}

\subsubsection{Файл с учетеными записями}
Внутри себя хранит следующую информацию:
\begin{itemize}
    \item UID
    \item Логин
    \item Пароль
    \item UID проездного билета
    \item ФИО
    \item Дату рождения
\end{itemize}

\subsection{Пример файла с информацией об останочных пунктах}
\begin{verbatim}
    22 serialization::archive 19 0 0 3 0 0 0 4 6 
    Vostok 1.00e+00 1.00e+00 0 0 0 0 
    254 6 Gikalo 3.00e+00 2.00e+00 0 0 162 6 
    Urucca 3.00e+00 2.00e+00 0 0
\end{verbatim}

\subsection{Структура выходных данных}
В качестве выходных данных используются те же файлы, что и во входных данных, однако при выходе из программы они перезаписываются, тем самым сохраняя внесенные пользователем изменения.

