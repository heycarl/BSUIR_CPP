%\section{Описание классов}
\section{Структурное проектирование}

Прежде чем описывать каждый класс в отдельности, стоит прояснить архитектуру программы в целом. При проектировании программы одной из основных задач была задача построить максимально гибкую и удобную для расширения систему. 

Для реализации задача был выбран подход, исользующий \textit{менеджеров} для управления множеством одинаковых объектов\cite{paterns}. Такой \textit{менеджер} предусмотрен для каждой значимой единицы и имеет соответствующие методы для управления этими объектами. Также менеджеры имеют обязательные для реализации функции сохранения и загрузки данных из базы данных (\textit{БД}).

Все мененджеры агрегированы в ядро программного продукта \textit{core}, который предоставляет доступ к функционалу менеджеров из вне.

Использование команд позволяет не только управлять изменением модели в обе стороны, но и при достаточной грануляции команд позволяет комбинировать их различным образом, что дает возможность реализовать карты способностей с различными поведениями а также преимущество повторного использования кода\cite{skaliar}. UML-диаграмма информационной системы представлена в Приложении Б (ГУИР.400201.002 PP). Листинг ИС приведен в приложении А.

\subsection{Реализация класса bus\_stop -- автобусной остановки}
\begin{verbatim}
class bus_stop {
	std::string get_options_string(); 
 - сериализация удобств остановки
	bus_stop() = default;  
 - конструктор по-умолчанию
	bus_stop(const std::string&, double, double); 
 - конструктор с параметрами
	UID get_uid() const; 
 - получение значения UID
	void set_uid(UID u); 
 - установка UID
	const std::string& get_name() const; 
 - получение названия
	void set_name(const std::string& n); 
 - установка названия
	std::string get_coords() const; 
 - получение координат
	void set_coords(double, double); 
 - установка координат
	void set_options(std::list<stop_options> stop_options); 
 - установка удобств остановки
	std::string view_existing_options(); 
 - получение всех существующих удобств
	std::string serialize_ui(); 
 - сериализация остановки для интерфейса
};
\end{verbatim}

\subsection{Реализация core -- ядра программы}
\begin{verbatim}
class core {
	static travel_cards_manager tcm;
 - объект менеджера проездных
	static routes_manager rm; 
 - объект менеджера маршрутов
	static bus_stops_manager bsm;
 - объект менеджера остановок
	static drivers_manager dm;
 - объект менеджера водителей
	static vehicles_manager vm;
 - объект менеджера ТС
	static void load();
 - метод загрузки данных из БД
	static void save();
 - метод загрузки данных в БД
};
\end{verbatim}

\subsection{Реализация класса person -- персоны}
\begin{verbatim}
class person {
	person(std::string fisrt_name, std::string last_name
 , std::string dob);
 - консткуртор с параметрами
	person() = default;
 - конструктор по-умолчанию
	virtual std::string serialize_ui();
 - виртуальный метод сериализатора для интерфейса 
	std::string get_first_last_name();
 - получение имени и фамилии
	UID get_uid() const;
 - получение UID
	static void ask_names_dob(std::string req, std::string& f_name,
 std::string& last_name, std::string& dob);
 - статический метод для запроса данных для регистрации
};
\end{verbatim}

\subsection{Реализация класса passenger -- пассажира}
\begin{verbatim}
class passenger : public person {
	passenger() = default;
	passenger(std::string f_name, std::string l_name, std::string dob);
	UID card;
 - UID проездного пассажира
	std::string serialize_ui() override;
	UID get_card();
 - получение UID проездного
};
\end{verbatim}

\subsection{Реализация класса driver -- водителя}
\begin{verbatim}
class driver : public person {
	driver() = default;
 - конструктор по-умолчанию
	driver(std::string fisrt_name, std::string last_name, 
 std::string dob, vehicle::vehicle_type v_type, date lic_exp);
 - конструктор с параметрами
	std::string serialize_ui() override
	{
		return person::serialize_ui()+serialize_license();
	};
 - сериализатор для интерфейса
	std::string serialize_license();
 - получение информации о лицензии
	const date& get_license_expiration() const;
 - получение даты истечения лицензии
	vehicle::vehicle_type get_license_type() const;
 - получение типа лицензии
};
\end{verbatim}

\subsection{Реализация класса drivers\_manager -- менеджера водителей}
\begin{verbatim}
class drivers_manager : public manager {
	driver& find_driver(UID);
 - поиск водителя по UID
	driver& add_driver(std::string fisrt_name, std::string last_name, 
 std::string dob, std::string v_type, std::string exp_date);
 - добавление нового водителя
	std::string serialize_all_drivers();
 - сериализация всех водителей
	bool check_if_driver_exists(UID);
 - проверка существования водителя
};
\end{verbatim}

\subsection{Реализация класса route -- маршрута}
\begin{verbatim}
class route {
	UID get_uid() const;
 - получение UID
	UID get_route_driver() const;
 - получение UID водителя
	void set_route_driver(UID driver);
 - установка UID водителя
	UID get_vehicle() const;
 - получение UID транспортного средства
	void set_vehicle(UID veh);
 - установка транспортного средства
	route() = default;
 - конструктор по-умолчани.
	route(UID vehicle, UID route_driver, std::string src, 
 std::string dst);
 - конструктор с параметрами
	std::string serialize_route();
 - сериализация направления рейса
	std::string serialize_full_route();
 - сериализация полного маршрута
	std::string serialize_stats();
 - сериализация статистики маршрута
	void increment_popularity();
 - увеличить популярность
	void delete_stop(uint8_t id);
 - удалить остановку из маршрута
	void add_stop(UID u, time_t arrival_time, bool need_to_stop);
 - добавить остановку в маршрут
	bool check_arrival_time(UID bus_stop_uid, time_t& arrival_time);
 - узнать время прибытия ТС на остановку
};
\end{verbatim}

\subsection{Реализация класса routes\_manager -- менеджера маршрутов}
\begin{verbatim}
class routes_manager : public manager{
	route& add_route(UID vehicle, UID driver, std::string src_point,
 std::string dst_point);
 - добавление маршрута
	route& find_route(UID);
 - поиск маршрута по UID
	std::list<route> get_routes_with_stop(UID stop_id);
 - получить список маршрутов с остановкой
	std::string serialize_all_routes_path();
 - сериализировать все направления маршуртов
	std::string serialize_all_routes_stats();
 - сериализировать всю статистику по маршрутам
};
\end{verbatim}

\subsection{Реализация класса travel\_card -- проездного}
\begin{verbatim}
class travel_card {
	UID uid;
	uint8_t remaining_trips;
 - количество поездок
	travel_card();
 - конструктор по-умолчанию
	~travel_card() = default;
 - деструктор
	explicit travel_card(UID);
 - конструктор с аргументом
	uint8_t get_remaining_trips() const;
 - получения количества поездок
	void set_remaining_trips(uint8_t remaining_trips);
 - установка поездок
	void decrement_trips();
 - уменьшение количества поездок
	UID get_uid() const;
 - получение UID
	void set_uid(UID u);
 - установка UID
};
\end{verbatim}

\subsection{Реализация класса travel\_cards\_manager -- менеджера проездных}
\begin{verbatim}
class travel_cards_manager : public manager {
	void create_if_not_exists(UID);
 - создать проездной, если не существует
	travel_card& find_travel_card(UID);
 - поиск проездного по UID
	static bool validator(travel_card&);
 - валидатор с функцией декремента поездок
};
\end{verbatim}

\subsection{Реализация класса uid\_generator -- генератора UID-ов}
\begin{verbatim}
class uid_generator {
   static UID generate(); - генерирует рандомный UID
};
\end{verbatim}

\subsection{Реализация класса user -- пользователя }
\begin{verbatim}
class user {
	std::string login;
	std::string password;
	std::string get_hash(std::string source_str);
 - в будущем функция будет использовать хэширование пароля
	user(std::string login, std::string password);
 - конструктор с параметрами
	user() = default;
 - конструктор по-умолчанию
	bool validate_credentials(std::string, std::string);
 - аутентификация
	static void ask_credentials(std::string, std::string&,
 std::string&);
};
\end{verbatim}

\subsection{Реализация класса user\_admin -- администратора}
\begin{verbatim}
class user_admin : public user {
	user_admin() = default;
 - конструктор по-умолчанию
	user_admin(const std::string&, const std::string&);
 - конструктор с параметрами
	void create_driver();
 - создание водителя
	void view_drivers();
 - просмотр водителей
	void modify_driver();
 - редактирование водителя
	void create_vehicle();
 - создание транспортного средства (ТС)
	void view_vehicles();
 - просмотр ТС
	void create_bus_stop();
 - создание остановки
	void view_bus_stops();
 - просмотр остановок
	void modify_bus_stop();
 -редактирование остановок
	void create_route();
 - создание маршрута
	void add_stop_to_route();
 - добавление остановки в маршрут
	void serialize_routes_stats();
 - сериализатор статистики маршрута
	void route_serialize_information();
 - сериализация информации о маршруте
};

\end{verbatim}

\subsection{Реализация класса user\_passenger -- пользователя пассажира}
\begin{verbatim}
class user_passenger : public user, public passenger {
	user_passenger() = default;
 - конструктор по-умолчанию
	user_passenger(std::string f_name, std::string l_name, 
 std::string dob, 
 std::string login, std::string password);
 - конструктор с параметрами
	void buy_trips();
 - покупка поездок
	void view_remaining_trips();
 - просмотр оставшихся поездок
	void enter_bus();
 - вход в ТС
	void get_arrival_time();
 - просмтр времени прибытия ТС на остановку
};
\end{verbatim}

\subsection{Реализация класса users\_manager -- менеджера пользователей}
\begin{verbatim}
class users_manager : public manager {
	std::list<user_passenger> l_passengers;
	std::list<user_admin> l_admins;
	user_passenger& sign_up_passenger();
 - регистрация пассажира
	user_admin& sign_up_admin();
 - регистрация администратора
	user_passenger& sign_in_passenger();
 - аутентификация пассажира
	user_admin& sign_in_admin();
 - аутентификация администратора
};
\end{verbatim}

\subsection{Реализация класса vehicle -- транспортного средства}
\begin{verbatim}
class vehicle {
	enum vehicle_type {
		bus,
		e_bus,
		tram
	}; - типы ТС
	static std::map<vehicle_type, std::string> vehicle_type_string;
 - соотношение ТС и названия
	UID uid;
	std::string registration_mark;
 - регистрационный номер
	vehicle_type type;
	uint8_t capacity;
 - вместительность
	vehicle() = default;
 - конструктор по-умолчанию
	UID get_uid() const;
 - просмотр UID
	const std::string& get_registration_mark() const;
 - просмотр регистрационного знака
	vehicle_type get_type() const;
 - просмотр типа ТС
	uint8_t get_capacity() const;
 - просмотр вместительности
	vehicle(std::string registration_mark, vehicle::vehicle_type type
 , uint8_t capacity);
	bool validate_card(travel_card&);
 - валидатор проездного
	virtual int get_travel_distance();
 - максимальное растояние к передвижению
	virtual std::string serialize_ui();
 - сериализация ТС для интерфейса
	static std::string serialize_vehicle_type(vehicle_type);
 - сериализация типа ТС
	static vehicle_type parse_vehicle_type(std::string);
 - парсинг типа ТС в тип ТС
	static std::string view_existing_types();
 - вывод существующих типов ТС
};
\end{verbatim}

\subsection{Реализация класса tram -- трамвая}
\begin{verbatim}
class tram : public vehicle {
	tram() = default;
	tram(std::string, uint8_t);
};
\end{verbatim}

\subsection{Реализация класса bus -- автобуса}
\begin{verbatim}
class bus : public vehicle {
	uint8_t fuel_bank_size;
 - вместительность топливного бака
	double fuel_consumption;
 - расход топлива
	bus() = default;
  - конструктор по-умолчанию
	bus(std::string, uint8_t, uint8_t, double);
  - конструктор c параметрами   
};
\end{verbatim}

\subsection{Реализация класса e\_bus -- электробуса}
\begin{verbatim}
class e_bus : public vehicle {
	double battery_consumption;
 - потребление заряда
	e_bus() = default;
 - конструктор по-умолчанию
	e_bus(std::string registration_mark, uint8_t capacity,
			double battery_consumption);
 - конструктор c параметрами 
};

\end{verbatim}

\subsection{Реализация класса vehicle\_manager -- менеджера ТС}
\begin{verbatim}
class vehicles_manager : public manager{
	std::list<bus> l_buses;
 - список автобусов
	std::list<e_bus> l_e_buses;
 - список электробусов
	std::list<tram> l_trams;
 - список трамваев
	bus& find_bus(UID);
 - поиск автобуса по UID
	bus& add_bus(std::string, uint8_t, uint8_t, double);
 - добавление нового автобуса
	std::string serialize_all_buses();
 - сериализация всех автобусов
	e_bus& find_e_bus(UID);
 - поиск электробуса по UID
	e_bus& add_e_bus(std::string, uint8_t, double);
 - добавление нового электробуса
	std::string serialize_all_e_buses();
 - сериализация всех электробусов
	tram& find_tram(UID);
 - поиск трамвая по UID
	tram& add_tram(std::string, uint8_t);
 - добавление нового трамвая
	std::string serialize_all_trams();
 - сериализация всех трамваев
	bool check_if_vehicle_exists(UID);
 - проверка на существование транспорта с UID
	vehicle::vehicle_type get_vehicle_type(UID);
 - возвращает тип ТС по его UID
	vehicle& get_vehicle_by_id(UID);
 - возвращает приведенное к vehicle ТС по UID 
};
\end{verbatim}

\subsection{Реализация класса controller -- управления ИС из терминала}
\begin{verbatim}
class controller {
	explicit controller(users_manager& u);
 - конструктор с аргументом менеджера пользователей
	void run();
 - запуск контроллера
	void admin_ui(user_admin& a);
 - UI администратора
	void passenger_ui(user_passenger& p);
 - UI пассажира
	users_manager& umanager;
 - объект менеджера пользователей
	struct func_admin {
		std::string description;
		std::_Mem_fn<void (user_admin::*)()> function;
	};
 - структура интерфейса администратора, содержащая 
 описание функции и саму функцию
	static std::map<int, struct func_admin> admin_functions;
 - карта индекса функции и структуры для администратора
	struct func_passenger {
		std::string description;
		std::_Mem_fn<void (user_passenger::*)()> function;
	};
  - структура интерфейса пользователя, содержащая 
 описание функции и саму функцию
	static std::map<int, struct func_passenger> passenger_functions;
 - карта индекса функции и структуры для пассажира
};
\end{verbatim}


\subsection{Реализация класса application -- основного класса приложения}
\begin{verbatim}
class date {
	uint16_t year = 0;
	uint8_t month = 0;
	uint8_t day = 0;
	date(uint16_t year, uint8_t month, uint8_t day);
 - конструктор с численными параметрами
	date(std::string);
 - конструктор с функцией парсинга даты из строки
	date() = default;
 - конструктор по-умолчанию
	std::string serialize_ui() const;
 - сериализация даты
	void set_date(uint16_t y, uint8_t m, uint8_t d);
 - установка с численными параметрами
	void set_date(std::tm);
 - установка из объекта времени
	void set_date(std::string);
 - установка с функцией парсинга даты из строки
	static bool date_parser(std::string& s, tm& date);
 - парсер даты из строки в объект времени
};
\end{verbatim}

\subsection{Реализация класса renderer -- вывод сообщений}
\begin{verbatim}
class renderer {
	static void render_boot_screen()
 - вывод заставки при загрузке ИС
	static void render_message(const std::string& msg)
 - вывод сообщения на экран
	static void render_error(const std::string& err)
 - вывод ошибки на экран
	static void render_headline(const std::string& headline)
 - вывод заголока
};
\end{verbatim}

\subsection{Реализация класса manager -- обязательные к реализации функции менеджера}
\begin{verbatim}
class manager {
public:
	virtual void save_db(const std::string&) = 0;
 - сохранение данных менеджера в БД
	virtual void load_db(const std::string&) = 0;
 - загрузка данных из БД в мененджер
};
\end{verbatim}

\subsection{Реализация класса application -- основного класса приложения}
\begin{verbatim}
class application {
	void run();
 - запуск приложения
	void quit();
 - завершение работы приложения
	~application();
 - декструктор
	users_manager umanager;
 - менеджер пользователей
};
\end{verbatim}