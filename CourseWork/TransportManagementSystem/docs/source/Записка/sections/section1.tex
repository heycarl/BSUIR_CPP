\section{Постановка задачи}

Программа должна иметь текстовый пользовательский интерфейс (\textit{TUI}) с необходимыми пунктами меню. Информация, использующаяся системой должна храниться в различных файлах (бинарных или текстовых), связанных определенным образом, и включать: информацию о пользователях, маршрутах, транспорте, остановках и проездных билетах. По запросу выдавать информацию о расписании, остановочных пунктах, оставшихся на проездном поездках, а так же статистику использования маршрута. Также необходимо создать возможность заполнять базу имеющихся остановок, маршрутах, водителях, пассажирах и транспорте. 

Разработать иерархию классов с использованием наследования. Разработать и использовать в программе классы контейнеров и итераторов (с использованием STL). Производить обработку исключительных ситуаций. 

В архитектуре ИС заложить архитектрную возможность для добавления функционала, новых видов транспорта. Предусмотреть два вида пользователей -- пассажир и администратор, а так же возможность создания их учетных записей.
Функционал администратора:
\begin{enumerate}
    \item Создавать нового водителя
    \item Просмотр имеющихся водителей
    \item Редактирование водителя
    \item Создание нового транспортного средства \textit{ТС}
    \item Просмотр имеющихся транспортных средств
    \item Создание остановочного пункта
    \item Просмотр имеющихся остановочных пунктов
    \item Редактирование остановочного пункта
    \item Создание нового маршута
    \item Добавление остановочного пункта в маршрут
    \item Просмотр статистики маршута
    \item Просмотр информации о маршруте
\end{enumerate}

Функционал пассажира:
\begin{enumerate}
    \item Покупка проездных билетов
    \item Просмотр количества оставшихся поездок
    \item Входить в транспортное средство
    \item Просмотр времени прибытия ТС на остановочном пункте
\end{enumerate}